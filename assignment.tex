\documentclass[10pt,a4paper]{article}
\usepackage{amssymb}
\usepackage{fullpage}
\usepackage{amsmath}

\title{Senior Nanoscience 2011 Assignment 2}
\date{}
\author{D. G. Wilcox \\
		309248035}

\begin{document}

\maketitle

\section*{Question 1}

\begin{itemize}
    % Explain the role of Brownian motion in molecular self assembly processes.
	\item[(a)] Brownian motion, although a common source of fluctuations and thus problems, can be used in self-assembly processes. Brownian motion causes particles to move due to thermal agitations, motion which when coupled with intelligently adherent substances can cause structure to form with extremely minimal energy input.

    As can be imagined, this means self-assembly through Brownian motion is extremely difficult. Designing a structure that will auto-"magically" collect the appropriate particles in the appropriate places as they randomly interact due to Brownian motion but don't allow unwanted particles to form on the structure is almost impossible to solve conceptually. In spite of this systems in nature utilise Brownian motion, resulting in the most efficent self-assembly processes known. Discovering these processes which are useful to us is always a good thing.
	\item[(b)] 
        \begin{itemize}
            \item The specification of a desired target structure.
            \item The design of a secondary structure of the taret molecule.
            \item The design of the primary structure.
        \end{itemize}
        This can be done either through:
        \begin{itemize}
            \item breaking the target structure into smaller units to make periodic lattices to form tile-based structures,
            \item folding the nanostructure out of a single long strand to form a folding structure, or
            \item abusing kinetics to allow for the dynamic assembly of the nanostructure.
        \end{itemize}
	\item[(c)] DNA nanotechnology treats DNA only as building blocks and not as a storage for genetic information. This is because DNA, with it's base-pairs, makes great building blocks, which have specific allowed and disallowed bindings. This lego like property is present in RNA and PNA as well, making them ideal candidates for the technologies developed in nanotechnology.
\end{itemize}

\section*{Question 2}

\begin{itemize}
	\item[(a)] The kinetic energy of an electron is given by $E_{k} = p^{2}/(2m)$ and the wavelength of an electron is given by $\lambda = h/p$. Substituting these together we get $\lambda = h/\sqrt{2mE_{k}}$. Taking into account relativistic effects gives us $\lambda = h/\sqrt{2mE_{k}(1+\frac{E_{k}}{2mc^{2}})}$.
	\item[(b)] If we sub values for $h, m, e $ and $c$ into the equation we get $\lambda = 3.7 \times 10^{-12}$.
\end{itemize}

\section*{Question 3}

\begin{itemize}
	\item[(a)]
	\item[(b)]
\end{itemize}

\end{document}
